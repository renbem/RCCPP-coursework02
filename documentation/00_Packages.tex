\usepackage[latin1,utf8]{inputenc}

%\usepackage[colorlinks=false]{hyperref}
				%colorlinks=true: Links von Referenzierungen in Farben, Nachteil: Bei Ausdruck auch in Farbe
				%colorlinks=false: Farbige Kaestchen und Ausdruck in Schwarz
\usepackage%
[colorlinks=true,
	citecolor=black,
	urlcolor=black,
	linkcolor=black]
{hyperref} %Alle Verlinkungen in schwarz.

%***hypersetup nur m\"oglich in Kombination mit hyperref!
\hypersetup{
	pdfauthor={Michael Ebner},
	pdftitle={RCCPP Coursework 01},
	%pdfkeywords={Reservoir Theory, Windkessel Model, Peripheral Arteries, Brachial Artery, Carotid Artery},
	%pdfsubject={Diploma Thesis} %Diplomarbeit, Seminararbeit etc
	%pdfcreationdate=20131204180532, 	%Ausgabe: Wednesday 04 December 2013 18:05:32
	%pdfmoddate=20131005 							%Ausgabe: Saturday 05 October 2013 00:00:00
	%Creation und modified date wird sowieso automatisch ausgegeben.
}

%***hypcap nur m\"oglich in Kombination mit hyperref!
\usepackage[all]{hypcap}  %Ohne dieser Zeile waere hyperref nur auf Caption unter Abbildung. Mit: ganze Abb. sichtbar

%***\texorpdfstring{''TEX text''}{''Bookmark Text''} %Interessant wenn Mathematische Formeln in \"Uberschriften
%***zB: \subsection{\texorpdfstring{$(h-h/2)$-Fehlersch\"atzer $\eta_\ell$ bzw. $\eta_\ell^*$}{(h-h/2)-Fehlersch\"atzer}}
%\newcommand{\texorpdfstring}[2]{#1} %Wenn Dokument ohne Hyperref erw\"unscht, dann wird Befehl \"uberschrieben --> keine Fehlermeldung

% Anpassen der Schriften des KOMA-Skripts:
%-----------------------------------------------------------------------------------------------------------------------
%Package noetig um gleiches font wie fuer ``article`` zu erhalten.
% \setkomafont{disposition}{\normalfont\bfseries}

% Aendert alle Gliederungsueberschriften (\part bis \subparagraph und \minisec) sowie Überschrift der Zusammenfassung
% Math-Umgebung wird in Ueberschrift fett angezeigt
\setkomafont{disposition}{\rmfamily\bfseries\boldmath}
 
% Aendert das Label (also den optionalen Eintrag einer \item-Anweisung) einer description-Umgebung 
%\setkomafont{descriptionlabel}{\rmfamily\bfseries\boldmath}   
%-----------------------------------------------------------------------------------------------------------------------
\usepackage[noadjust]{cite} %normal liefert \cite{} ein automatisches Leerzeichen vor Referenz; noadjust verhindert dies. Bsp: (\cite{test}) =( [2]) --> ([2])
\usepackage[ngerman,english]{babel}
\usepackage[T1]{fontenc}	%Damit wird zB auch \textsc{Matlab} in Kapitaelchen in Ueberschriften angezeigt
				%unbedingt "cm-super" installieren! sonst Schrift gerastert(pixelig) (alternative lmodern-package)
				%\usepackage{lmodern}		%mit [T1]{fontenc} Schrift "`pixelig"' (gerastert) -> lmodern verhindert dies.
\usepackage{ifthen,amsmath,amssymb,latexsym}
\usepackage{fancyhdr}
\usepackage{psfrag}
\usepackage{graphicx}
\usepackage[format=plain,justification=centering]{caption}
\usepackage{subfig}
% \captionsetup[subfloat]{captionskip=10pt}
\usepackage{mdframed}
% \usepackage{pgfplots}

\usepackage{listings}

\definecolor{mygreen}{rgb}{0,0.6,0}
\definecolor{mygray}{rgb}{0.5,0.5,0.5}
\definecolor{mymauve}{rgb}{0.58,0,0.82}
\definecolor{mybackgroundgray}{rgb}{0.9,0.9,0.9}
\definecolor{gray}{rgb}{0.75,0.75,0.75}
\definecolor{lightgray}{rgb}{0.95,0.95,0.95}

\lstdefinestyle{myCppStyle}{ %
	backgroundcolor=\color{mybackgroundgray},   % choose the background color; you must add \usepackage{color} or \usepackage{xcolor}
	basicstyle=\ttfamily\footnotesize,        % the size of the fonts that are used for the code
%	breakatwhitespace=false,         % sets if automatic breaks should only happen at whitespace
%	breaklines=true,                 % sets automatic line breaking
%	captionpos=b,                    % sets the caption-position to bottom
	commentstyle=\color{mygreen},    % comment style
%	deletekeywords={...},            % if you want to delete keywords from the given language
%	escapeinside={\%*}{*)},          % if you want to add LaTeX within your code
	extendedchars=true,              % lets you use non-ASCII characters; for 8-bits encodings only, does not work with UTF-8
%	frame=single,                    % adds a frame around the code
	keepspaces=true,                 % keeps spaces in text, useful for keeping indentation of code (possibly needs columns=flexible)
	keywordstyle=\color{blue},       % keyword style
	language=C++,                 % the language of the code
	morekeywords={pragma,omp,parallel,ifdef,endif},            % if you want to add more keywords to the set
	numbers=left,                    % where to put the line-numbers; possible values are (none, left, right)
%	numbersep=5pt,                   % how far the line-numbers are from the code
	numberstyle=\tiny\color{gray}, % the style that is used for the line-numbers
	rulecolor=\color{black},         % if not set, the frame-color may be changed on line-breaks within not-black text (e.g. comments (green here))
	showspaces=false,                % show spaces everywhere adding particular underscores; it overrides 'showstringspaces'
	showstringspaces=false,          % underline spaces within strings only
	showtabs=false,                  % show tabs within strings adding particular underscores
	stepnumber=1,                    % the step between two line-numbers. If it's 1, each line will be numbered
	stringstyle=\color{mymauve},     % string literal style
	tabsize=4,                       % sets default tabsize to 2 spaces
	%title=\lstname                   % show the filename of files included with \lstinputlisting; also try caption instead of title
}


\usepackage{cleveref}
\crefname{algorithm}{algorithm}{algorithms}

%***TikZ

\usepackage{fp,tikz}       %fixed point arithmetic for tex, tikz f\"ur Befehl foreach
% \usetikzlibrary{snakes}	%for underbraces
\usetikzlibrary{patterns,calc,intersections,arrows}

\newcommand*{\rechterWinkel}[3]{% #1 = point, #2 = start angle, #3 = radius
   \draw[shift={(#2:#3)}] (#1) arc[start angle=#2, delta angle=90, radius = #3];
   \fill[shift={(#2+45:#3/2)}] (#1) circle[radius=1.25\pgflinewidth];
}

\usepackage{enumitem}

\usepackage[
	separate-uncertainty=true,
	scientific-notation=false,
	%multi-part-units=single,
	list-units=repeat,
	per-mode = symbol
	]
	{siunitx}
%[ %
%	separate-uncertainity=true
%    ,uncertainity-separator={\pm}
        %mode = text,
%   ,multi-part-units = brackets
%	,decimalsymbol=comma
%]    

\sisetup{binary-units=true,separate-uncertainty}

% \usepackage{layout}
% \setlength{\marginparwidth}{0pt}

%\usepackage{pifont} %F\"ur q.e.d


%\setlength\parindent{0pt} %Festlegen des Absatzeinzuges

% %---F\"ur Symbolverzeichnis, Abk\"urzungsverz. und Glossaries:
% \usepackage[ngerman]{translator}
% \usepackage[
% 	nonumberlist, %keine Seitenzahlen anzeigen
% 	acronym,      %ein Abkuerzungsverzeichnis erstellen
% 	toc]          %Eintraege im Inhaltsverzeichnis
% 	%section]      %im Inhaltsverzeichnis auf section-Ebene erscheinen %nicht bei scrreprt
% 	{glossaries}

%\numberwithin{equation}{section} %koppelt Counter equation an chapter
%\numberwithin{figure}{section} %koppelt Counter figure an chapter


%---Tabellen (tabular) Formatierung:
%-Der Abstand zwischen den Zeilen einer Tabelle kann über folgenden Befehl angepasst werden:
%-\renewcommand{\arraystretch}{Faktor}
\renewcommand{\arraystretch}{1.2}
\usepackage{colortbl} %Paket notwending, dass arrayrulecolor ge\"andert werden kann
\arrayrulecolor[gray]{0.8}  %Linienfarbe der Tabelle
\usepackage{array}
\newcolumntype{L}[1]{>{\raggedright\let\newline\\\arraybackslash\hspace{0pt}}m{#1}}
\newcolumntype{C}[1]{>{\centering\let\newline\\\arraybackslash\hspace{0pt}}m{#1}}
\newcolumntype{R}[1]{>{\raggedleft\let\newline\\\arraybackslash\hspace{0pt}}m{#1}}

%---Formatierung bei zB enumerate Umgebung: Einfach \begin{enumerate}[(a)] oder ..[I] eingeben
%\usepackage{paralist}

\usepackage{multirow}